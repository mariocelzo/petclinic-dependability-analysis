\documentclass[aspectratio=169,11pt]{beamer}

% ============================================
% TEMA E COLORI
% ============================================
\usetheme{Madrid}
\usecolortheme{whale}

% Colori personalizzati UniSA
\definecolor{unisablue}{RGB}{0,51,102}
\definecolor{unisagold}{RGB}{204,153,0}
\setbeamercolor{structure}{fg=unisablue}
\setbeamercolor{title}{fg=white,bg=unisablue}
\setbeamercolor{frametitle}{fg=white,bg=unisablue}

% ============================================
% PACCHETTI
% ============================================
\usepackage[utf8]{inputenc}
\usepackage[T1]{fontenc}
\usepackage{graphicx}
\usepackage{booktabs}
\usepackage{hyperref}
\usepackage{tikz}
\usepackage{fontawesome5}
\usepackage{xcolor}

% ============================================
% INFO DOCUMENTO
% ============================================
\title[PetClinic Dependability]{Dependability Analysis of Spring PetClinic}
\subtitle{Software Dependability Course Project}
\author[M. Celzo]{Mario Celzo\\{\small Matricola: 0512118665}}
\institute[UniSA]{Università degli Studi di Salerno\\Dipartimento di Informatica}
\date{Anno Accademico 2024-2025}

% ============================================
% INIZIO DOCUMENTO
% ============================================
\begin{document}

% ------------------------------------------
% SLIDE 1: Titolo
% ------------------------------------------
\begin{frame}
    \titlepage
\end{frame}

% ------------------------------------------
% SLIDE 2: Agenda
% ------------------------------------------
\begin{frame}{Agenda}
    \tableofcontents
\end{frame}

% ============================================
\section{Introduzione}
% ============================================

% ------------------------------------------
% SLIDE 3: Obiettivi del Progetto
% ------------------------------------------
\begin{frame}{Obiettivi del Progetto}
    \begin{block}{Obiettivo Principale}
        Analizzare la \textbf{dependability} di Spring PetClinic attraverso 9 criteri di valutazione, documentando i miglioramenti ottenuti con un approccio Before/After.
    \end{block}
    
    \vspace{0.5cm}
    
    \begin{columns}
        \column{0.5\textwidth}
        \textbf{Progetto Analizzato:}
        \begin{itemize}
            \item Spring PetClinic
            \item Java 21 + Spring Boot 4.0
            \item Applicazione web reference
        \end{itemize}
        
        \column{0.5\textwidth}
        \textbf{Approccio:}
        \begin{itemize}
            \item Fork del repository originale
            \item Analisi baseline (Before)
            \item Enhancement sistematico (After)
        \end{itemize}
    \end{columns}
\end{frame}

% ------------------------------------------
% SLIDE 4: I 9 Criteri di Valutazione
% ------------------------------------------
\begin{frame}{I 9 Criteri di Valutazione}
    \begin{table}
        \centering
        \small
        \begin{tabular}{clc}
            \toprule
            \textbf{\#} & \textbf{Criterio} & \textbf{Tool} \\
            \midrule
            1 & CI/CD Pipeline & GitHub Actions \\
            2 & Static Code Analysis & SonarCloud \\
            3 & Docker Image & DockerHub \\
            4 & Container Testing & Health Check \\
            5 & Code Coverage & JaCoCo \\
            6 & Mutation Testing & PITest \\
            7 & Performance Benchmarks & JMH \\
            8 & Test Generation & Randoop \\
            9 & Security Analysis & SpotBugs + OWASP \\
            \bottomrule
        \end{tabular}
    \end{table}
\end{frame}

% ============================================
\section{Baseline Analysis}
% ============================================

% ------------------------------------------
% SLIDE 5: Stato Iniziale del Progetto
% ------------------------------------------
\begin{frame}{Stato Iniziale: Spring PetClinic Originale}
    \begin{table}
        \centering
        \small
        \begin{tabular}{lcc}
            \toprule
            \textbf{Aspetto} & \textbf{Stato Originale} & \textbf{Valutazione} \\
            \midrule
            CI/CD Pipeline & 1 workflow base & \textcolor{orange}{Parziale} \\
            Code Coverage & Non configurato & \textcolor{red}{Assente} \\
            Mutation Testing & Non configurato & \textcolor{red}{Assente} \\
            SonarCloud & Non integrato & \textcolor{red}{Assente} \\
            Security Analysis & Non configurato & \textcolor{red}{Assente} \\
            JMH Benchmarks & Non implementato & \textcolor{red}{Assente} \\
            Docker & Dockerfile base & \textcolor{orange}{Parziale} \\
            Test Generation & Non implementato & \textcolor{red}{Assente} \\
            \bottomrule
        \end{tabular}
    \end{table}
    
    \begin{alertblock}{Conclusione Baseline}
        Il progetto originale mancava della maggior parte degli strumenti di analisi della dependability.
    \end{alertblock}
\end{frame}

% ============================================
\section{Risultati: Before vs After}
% ============================================

% ------------------------------------------
% SLIDE 6: Executive Summary
% ------------------------------------------
\begin{frame}{Executive Summary: Before vs After}
    \begin{table}
        \centering
        \small
        \begin{tabular}{lccc}
            \toprule
            \textbf{Metrica} & \textbf{Before} & \textbf{After} & \textbf{Miglioramento} \\
            \midrule
            CI/CD Workflows & 1 & 4 & \textcolor{green!60!black}{+300\%} \\
            Code Coverage & N/A & 91.9\% & \textcolor{green!60!black}{Nuova metrica} \\
            Mutation Score & N/A & 85\% & \textcolor{green!60!black}{Nuova metrica} \\
            SonarCloud Rating & N/A & AAA & \textcolor{green!60!black}{Triple-A} \\
            Security Issues & Unknown & 0 High & \textcolor{green!60!black}{Verificato} \\
            Test Count & 39 & 556 & \textcolor{green!60!black}{+1326\%} \\
            JMH Benchmarks & 0 & 8 & \textcolor{green!60!black}{Nuova suite} \\
            \bottomrule
        \end{tabular}
    \end{table}
\end{frame}

% ------------------------------------------
% SLIDE 7: CI/CD Pipeline
% ------------------------------------------
\begin{frame}{Criterio 1: CI/CD Pipeline}
    \begin{columns}
        \column{0.45\textwidth}
        \begin{block}{Before}
            \begin{itemize}
                \item 1 workflow base
                \item Solo build Maven
                \item Nessun test automatico
                \item Nessun deploy
            \end{itemize}
        \end{block}
        
        \column{0.45\textwidth}
        \begin{exampleblock}{After}
            \begin{itemize}
                \item 4 workflows completi
                \item Build + Test + Coverage
                \item Docker auto-push
                \item PDF report auto-generato
            \end{itemize}
        \end{exampleblock}
    \end{columns}
    
    \vspace{0.5cm}
    
    \begin{table}
        \centering
        \small
        \begin{tabular}{ll}
            \toprule
            \textbf{Workflow} & \textbf{Funzione} \\
            \midrule
            \texttt{ci.yml} & Build, test, coverage, mutation, security \\
            \texttt{docker.yml} & Build image, push DockerHub, health check \\
            \texttt{sonarcloud.yml} & Analisi qualità codice \\
            \texttt{latex-pdf.yml} & Generazione report PDF \\
            \bottomrule
        \end{tabular}
    \end{table}
\end{frame}

% ------------------------------------------
% SLIDE 8: SonarCloud
% ------------------------------------------
\begin{frame}{Criterio 2: SonarCloud Analysis}
    \begin{columns}
        \column{0.45\textwidth}
        \begin{block}{Before}
            \centering
            \textbf{Non integrato}
            
            \vspace{0.3cm}
            Nessuna metrica di qualità disponibile
        \end{block}
        
        \column{0.45\textwidth}
        \begin{exampleblock}{After: Triple-A Rating}
            \begin{table}
                \centering
                \small
                \begin{tabular}{lc}
                    Bugs & 0 \\
                    Vulnerabilities & 0 \\
                    Code Smells & 48 \\
                    Coverage & 91.9\% \\
                \end{tabular}
            \end{table}
        \end{exampleblock}
    \end{columns}
    
    \vspace{0.5cm}
    
    \begin{center}
        \Large
        \textbf{Reliability: A} \quad \textbf{Security: A} \quad \textbf{Maintainability: A}
    \end{center}
\end{frame}

% ------------------------------------------
% SLIDE 9: Code Coverage (JaCoCo)
% ------------------------------------------
\begin{frame}{Criterio 5: Code Coverage (JaCoCo)}
    \begin{columns}
        \column{0.5\textwidth}
        \begin{table}
            \centering
            \small
            \begin{tabular}{lcc}
                \toprule
                \textbf{Metrica} & \textbf{Before} & \textbf{After} \\
                \midrule
                Instructions & N/A & 90\% \\
                Branches & N/A & 84\% \\
                Lines & N/A & 91.9\% \\
                Methods & N/A & 85\% \\
                Classes & N/A & 91\% \\
                \bottomrule
            \end{tabular}
        \end{table}
        
        \column{0.5\textwidth}
        \begin{table}
            \centering
            \small
            \begin{tabular}{lc}
                \toprule
                \textbf{Package} & \textbf{Coverage} \\
                \midrule
                model & 98\% \\
                owner & 93\% \\
                vet & 91\% \\
                visit & 89\% \\
                system & 86\% \\
                \bottomrule
            \end{tabular}
        \end{table}
    \end{columns}
    
    \vspace{0.3cm}
    
    \begin{alertblock}{Risultato}
        Coverage complessiva del \textbf{91.9\%} - ben oltre la soglia standard del 80\%.
    \end{alertblock}
\end{frame}

% ------------------------------------------
% SLIDE 10: Mutation Testing (PITest)
% ------------------------------------------
\begin{frame}{Criterio 6: Mutation Testing (PITest)}
    \begin{columns}
        \column{0.5\textwidth}
        \begin{block}{Cos'è il Mutation Testing?}
            Introduce \textit{mutazioni} nel codice per verificare se i test le rilevano. Un test efficace deve ``uccidere'' le mutazioni.
        \end{block}
        
        \column{0.5\textwidth}
        \begin{table}
            \centering
            \small
            \begin{tabular}{lc}
                \toprule
                \textbf{Metrica} & \textbf{Valore} \\
                \midrule
                Mutations Generated & 1,003 \\
                Mutations Killed & 853 \\
                \textbf{Mutation Score} & \textbf{85\%} \\
                Test Strength & 94\% \\
                \bottomrule
            \end{tabular}
        \end{table}
    \end{columns}
    
    \vspace{0.3cm}
    
    \begin{table}
        \centering
        \small
        \begin{tabular}{lcc}
            \toprule
            \textbf{Mutator} & \textbf{Generated} & \textbf{Kill Rate} \\
            \midrule
            Conditional Boundary & 143 & 90\% \\
            Negate Conditionals & 187 & 88\% \\
            Return Values & 234 & 85\% \\
            \bottomrule
        \end{tabular}
    \end{table}
\end{frame}

% ------------------------------------------
% SLIDE 11: Docker & Container Testing
% ------------------------------------------
\begin{frame}{Criteri 3-4: Docker Image \& Container Testing}
    \begin{columns}
        \column{0.45\textwidth}
        \begin{block}{Before}
            \begin{itemize}
                \item Dockerfile base
                \item Nessun health check
                \item Push manuale
                \item No CI integration
            \end{itemize}
        \end{block}
        
        \column{0.45\textwidth}
        \begin{exampleblock}{After}
            \begin{itemize}
                \item Multi-stage build
                \item Health check integrato
                \item Auto-push su DockerHub
                \item Test container in CI
            \end{itemize}
        \end{exampleblock}
    \end{columns}
    
    \vspace{0.5cm}
    
    \begin{table}
        \centering
        \small
        \begin{tabular}{ll}
            \toprule
            \textbf{Aspetto} & \textbf{Dettaglio} \\
            \midrule
            Repository & \texttt{wario03/petclinic-dependability} \\
            Health Endpoint & \texttt{/actuator/health} \\
            CI Testing & Container avviato e testato automaticamente \\
            Total Pulls & 144+ \\
            \bottomrule
        \end{tabular}
    \end{table}
\end{frame}

% ------------------------------------------
% SLIDE 12: JMH Performance Benchmarks
% ------------------------------------------
\begin{frame}{Criterio 7: Performance Benchmarks (JMH)}
    \begin{block}{Prima: Nessun Benchmark}
        Nessuna baseline di performance documentata.
    \end{block}
    
    \vspace{0.3cm}
    
    \begin{exampleblock}{Dopo: 8 Benchmark Implementati}
        \begin{table}
            \centering
            \small
            \begin{tabular}{lcr}
                \toprule
                \textbf{Operation} & \textbf{Avg Time (ms/op)} & \textbf{Status} \\
                \midrule
                OwnerCreation & 0.00012 & Optimal \\
                PetCreation & 0.00008 & Optimal \\
                VisitCreation & 0.00015 & Optimal \\
                VetLookup & 0.00023 & Optimal \\
                OwnerSearch & 0.00089 & Optimal \\
                \bottomrule
            \end{tabular}
        \end{table}
    \end{exampleblock}
    
    \textbf{Key Finding:} Tutte le operazioni completano in meno di 1ms.
\end{frame}

% ------------------------------------------
% SLIDE 13: Test Generation (Randoop)
% ------------------------------------------
\begin{frame}{Criterio 8: Automated Test Generation (Randoop)}
    \begin{columns}
        \column{0.5\textwidth}
        \begin{table}
            \centering
            \begin{tabular}{lcc}
                \toprule
                \textbf{Metrica} & \textbf{Before} & \textbf{After} \\
                \midrule
                Manual Tests & 39 & 39 \\
                Generated Tests & 0 & 517 \\
                \textbf{Total Tests} & \textbf{39} & \textbf{556} \\
                Lines of Code & $\sim$2K & $\sim$30K \\
                \bottomrule
            \end{tabular}
        \end{table}
        
        \column{0.5\textwidth}
        \begin{alertblock}{Improvement}
            \centering
            \Huge\textbf{+1326\%}
            
            \normalsize Test Cases
        \end{alertblock}
    \end{columns}
    
    \vspace{0.5cm}
    
    Randoop ha generato automaticamente 517 regression tests, tutti eseguibili e passanti.
\end{frame}

% ------------------------------------------
% SLIDE 14: Security Analysis
% ------------------------------------------
\begin{frame}{Criterio 9: Security Analysis}
    \begin{columns}
        \column{0.45\textwidth}
        \begin{block}{Before}
            \begin{itemize}
                \item Nessuno scan di sicurezza
                \item Vulnerabilità sconosciute
                \item No dependency check
            \end{itemize}
        \end{block}
        
        \column{0.45\textwidth}
        \begin{exampleblock}{After}
            \begin{itemize}
                \item SpotBugs + FindSecBugs
                \item OWASP Dependency-Check
                \item GitHub Secrets per API keys
            \end{itemize}
        \end{exampleblock}
    \end{columns}
    
    \vspace{0.5cm}
    
    \begin{table}
        \centering
        \begin{tabular}{lcc}
            \toprule
            \textbf{Severity} & \textbf{Before} & \textbf{After} \\
            \midrule
            Critical & Unknown & \textcolor{green!60!black}{0} \\
            High & Unknown & \textcolor{green!60!black}{0} \\
            Medium & Unknown & 3 \\
            Low (Info) & Unknown & 5 \\
            \bottomrule
        \end{tabular}
    \end{table}
\end{frame}

% ============================================
\section{Conclusioni}
% ============================================

% ------------------------------------------
% SLIDE 15: Riepilogo Finale
% ------------------------------------------
\begin{frame}{Riepilogo: Tutti i 9 Criteri Superati}
    \begin{table}
        \centering
        \small
        \begin{tabular}{clcc}
            \toprule
            \textbf{\#} & \textbf{Criterio} & \textbf{Stato} & \textbf{Evidenza} \\
            \midrule
            1 & CI/CD Pipeline & \textcolor{green!60!black}{\faCheckCircle} & 4 workflows \\
            2 & SonarCloud & \textcolor{green!60!black}{\faCheckCircle} & AAA Rating \\
            3 & Docker Image & \textcolor{green!60!black}{\faCheckCircle} & DockerHub push \\
            4 & Container Test & \textcolor{green!60!black}{\faCheckCircle} & Health Check OK \\
            5 & JaCoCo Coverage & \textcolor{green!60!black}{\faCheckCircle} & 91.9\% \\
            6 & PITest Mutation & \textcolor{green!60!black}{\faCheckCircle} & 85\% score \\
            7 & JMH Benchmarks & \textcolor{green!60!black}{\faCheckCircle} & 8 benchmarks \\
            8 & Randoop Tests & \textcolor{green!60!black}{\faCheckCircle} & 517 tests \\
            9 & Security Scan & \textcolor{green!60!black}{\faCheckCircle} & 0 Critical/High \\
            \bottomrule
        \end{tabular}
    \end{table}
\end{frame}

% ------------------------------------------
% SLIDE 16: Lessons Learned
% ------------------------------------------
\begin{frame}{Lessons Learned}
    \begin{enumerate}
        \item \textbf{Baseline First}: Stabilire metriche di partenza prima di qualsiasi modifica è essenziale per dimostrare i miglioramenti.
        
        \vspace{0.3cm}
        
        \item \textbf{Automation Pays Off}: L'investimento iniziale in CI/CD ripaga con quality checks consistenti su ogni commit.
        
        \vspace{0.3cm}
        
        \item \textbf{Security Early}: Integrare l'analisi di sicurezza fin dall'inizio previene problemi costosi.
        
        \vspace{0.3cm}
        
        \item \textbf{Test Quality > Quantity}: Il mutation testing (85\%) valida l'efficacia dei test meglio della sola coverage (91.9\%).
    \end{enumerate}
\end{frame}

% ------------------------------------------
% SLIDE 17: Links & Resources
% ------------------------------------------
\begin{frame}{Links \& Resources}
    \begin{table}
        \centering
        \begin{tabular}{ll}
            \toprule
            \textbf{Resource} & \textbf{URL} \\
            \midrule
            GitHub Repository & github.com/mariocelzo/petclinic-dependability-analysis \\
            SonarCloud & sonarcloud.io \\
            DockerHub & hub.docker.com/r/wario03/petclinic-dependability \\
            \bottomrule
        \end{tabular}
    \end{table}
    
    \vspace{1cm}
    
    \centering
    \Large\textbf{Grazie per l'attenzione!}
    
    \vspace{0.5cm}
    
    \normalsize Domande?
\end{frame}

\end{document}
