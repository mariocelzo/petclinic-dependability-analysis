\section{Conclusions}
\label{sec:conclusions}

\subsection{Summary}

This project conducted a comprehensive dependability analysis of the Spring PetClinic 
application, evaluating multiple quality criteria using industry-standard tools 
and methodologies. The analysis demonstrates that Spring PetClinic is a well-engineered 
application with excellent code quality metrics.

\subsection{Key Findings}

\subsubsection{Completed Achievements}

The analysis successfully accomplished the following:

\begin{enumerate}
    \item \textbf{Established CI/CD Pipeline}: Three automated GitHub Actions workflows 
    (CI, Docker, SonarCloud) ensure continuous quality monitoring with 100\% build success rate.
    
    \item \textbf{Verified Excellent Coverage}: 91.9\% test coverage significantly exceeds 
    the 80\% industry target, with 44 tests all passing.
    
    \item \textbf{Confirmed Zero Vulnerabilities}: SonarCloud security analysis found 
    no security vulnerabilities or bugs in the codebase.
    
    \item \textbf{Achieved Triple-A Rating}: SonarCloud Quality Gate passed with A ratings 
    for Security, Reliability, and Maintainability.
    
    \item \textbf{Containerized Application}: Created production-ready Docker image 
    with multi-stage build, non-root user, and health checks.
    
    \item \textbf{Automated Publishing}: Docker images automatically pushed to DockerHub 
    with semantic tagging (latest, branch, commit SHA).
    
    \item \textbf{Documented Troubleshooting}: Comprehensive documentation of all challenges 
    encountered and their solutions.
\end{enumerate}

\subsubsection{Quality Assessment}

\begin{table}[h]
\centering
\caption{Final Quality Assessment (28 November 2025)}
\label{tab:final-assessment}
\begin{tabular}{lccc}
\toprule
\textbf{Criterion} & \textbf{Target} & \textbf{Achieved} & \textbf{Status} \\
\midrule
CI/CD Pipeline & Pass & 100\% success & \checkmark \\
Code Quality (SonarCloud) & A Grade & A Grade & \checkmark \\
Docker Image & Published & On DockerHub & \checkmark \\
Test Coverage & > 80\% & 91.9\% & \checkmark \\
Security Vulnerabilities & 0 Critical & 0 Found & \checkmark \\
Bugs & 0 & 0 Found & \checkmark \\
Code Smells & < 50 & 23 (Minor) & \checkmark \\
Duplications & < 3\% & 0.0\% & \checkmark \\
\bottomrule
\end{tabular}
\end{table}

\subsection{Challenges and Solutions}

The project encountered several technical challenges that provided valuable learning experiences:

\begin{enumerate}
    \item \textbf{Java Version Mismatch}: Project required Java 21, resolved by installing 
    Eclipse Temurin distribution and configuring environment variables.
    
    \item \textbf{OWASP Dependency-Check Failure}: NVD API now requires authentication 
    (since December 2023). Resolved by skipping the check in CI and relying on SonarCloud's 
    security analysis.
    
    \item \textbf{SonarCloud Authorization}: Required adding \texttt{sonar.organization} 
    property to \texttt{pom.xml}.
    
    \item \textbf{GitHub Actions Debugging}: CI failures with truncated logs required 
    local reproduction to identify root causes.
\end{enumerate}

\subsection{Lessons Learned}

\subsubsection{Technical Insights}

\begin{enumerate}
    \item \textbf{Local Testing is Essential}: Always reproduce CI failures locally to see 
    complete error messages.
    
    \item \textbf{External Dependencies Change}: NVD API authentication requirement shows 
    that external service dependencies can break without warning.
    
    \item \textbf{SonarCloud Provides Comprehensive Analysis}: A single tool can assess 
    security, reliability, maintainability, coverage, and duplications.
    
    \item \textbf{Multi-stage Docker Builds}: Separating build and runtime stages produces 
    smaller, more secure images.
    
    \item \textbf{Documentation Value}: Detailed troubleshooting documentation saves time 
    when similar issues arise.
\end{enumerate}

\subsubsection{Process Insights}

\begin{enumerate}
    \item \textbf{Early Integration}: Integrating analysis tools into CI/CD early prevents 
    quality regression.
    
    \item \textbf{Incremental Improvement}: Fixing issues incrementally is more manageable 
    than addressing everything at once.
    
    \item \textbf{Documentation Importance}: Comprehensive documentation of rationale for 
    fixes aids future maintainers.
    
    \item \textbf{Time Estimation}: Analysis tasks take longer than expected. PITest 
    execution, test generation review, and manual fixes consumed significant time.
\end{enumerate}

\subsection{Limitations}

\subsubsection{Scope Limitations}

\begin{itemize}
    \item \textbf{Application Size}: Small application may not expose issues present in 
    larger systems
    
    \item \textbf{Load Testing}: Did not perform comprehensive load/stress testing
    
    \item \textbf{User Acceptance}: Focused on technical quality, not user experience
    
    \item \textbf{Production Environment}: Analysis conducted in development environment
\end{itemize}

\subsubsection{Methodological Limitations}

\begin{itemize}
    \item \textbf{Tool Accuracy}: Automated tools may produce false positives/negatives
    
    \item \textbf{Metric Selection}: Chosen metrics may not capture all quality aspects
    
    \item \textbf{Generalizability}: Results specific to Spring Boot/Java ecosystem
\end{itemize}

\subsection{Future Work}

\subsubsection{Short-term Extensions}

\begin{enumerate}
    \item \textbf{Frontend Testing}: Add Selenium/Cypress tests for UI coverage
    
    \item \textbf{Integration Testing}: Expand integration test suite for controller-service-repository flow
    
    \item \textbf{Load Testing}: Conduct JMeter/Gatling load tests
    
    \item \textbf{Chaos Engineering}: Test resilience under failure scenarios
\end{enumerate}

\subsubsection{Long-term Research Directions}

\begin{enumerate}
    \item \textbf{ML-based Test Generation}: Investigate machine learning approaches 
    for smarter test generation
    
    \item \textbf{Continuous Monitoring}: Implement runtime monitoring in production
    
    \item \textbf{Comparative Studies}: Compare results across different Spring Boot 
    applications
    
    \item \textbf{Cost-Benefit Analysis}: Quantify ROI of different testing strategies
\end{enumerate}

\subsection{Recommendations}

\subsubsection{For Project Teams}

\begin{enumerate}
    \item Integrate analysis tools early in development lifecycle
    \item Set quality gates in CI/CD pipeline
    \item Prioritize mutation testing over just coverage
    \item Regular security scanning (weekly/monthly)
    \item Review and refine generated tests
\end{enumerate}

\subsubsection{For Tool Users}

\begin{enumerate}
    \item \textbf{JaCoCo}: Focus on branch coverage, not just line coverage
    \item \textbf{PITest}: Start with critical packages, expand gradually
    \item \textbf{JMH}: Isolate benchmarks, use adequate warmup
    \item \textbf{EvoSuite}: Treat as supplement, not replacement for manual tests
    \item \textbf{OWASP}: Automate scans, update dependencies regularly
\end{enumerate}

\subsection{Final Remarks}

Software dependability is multidimensional and requires systematic evaluation across 
multiple quality attributes. This project demonstrated that:

\begin{itemize}
    \item Automated tools enable comprehensive analysis at scale
    \item Combining multiple tools provides complementary insights
    \item Continuous improvement requires both automation and human judgment
    \item Documentation and reproducibility are essential for long-term value
\end{itemize}

The techniques and findings from this analysis are applicable beyond Spring PetClinic 
to other Spring Boot applications and, with appropriate tool selection, to software 
projects in general.

\vspace{1cm}

\noindent\textbf{Project Repository}:  
\url{https://github.com/YOUR_USERNAME/petclinic-dependability-analysis}

\vspace{0.5cm}

\noindent\textbf{DockerHub Image}:  
\url{https://hub.docker.com/r/YOUR_DOCKERHUB/petclinic-dependability}

\vspace{0.5cm}

\noindent\textbf{SonarCloud Dashboard}:  
\url{https://sonarcloud.io/dashboard?id=YOUR_PROJECT}
