\section{Setup e Configurazione Progetto}
\label{sec:setup-configuration}

Questa sezione documenta le operazioni di setup e configurazione eseguite per preparare l'ambiente di analisi di dependability del progetto Spring PetClinic.

\subsection{Repository GitHub}
\label{subsec:github-repository}

\subsubsection{Creazione Repository}
Il progetto è stato configurato con un repository GitHub dedicato:

\begin{itemize}
    \item \textbf{Owner}: mariocelzo
    \item \textbf{Repository}: petclinic-dependability-analysis
    \item \textbf{URL}: \url{https://github.com/mariocelzo/petclinic-dependability-analysis}
    \item \textbf{Visibilità}: Public
    \item \textbf{Branch principale}: main
\end{itemize}

\subsubsection{Configurazione Remote Git}
Per mantenere la possibilità di sincronizzare con il repository upstream di Spring Projects, sono stati configurati due remote:

\begin{lstlisting}[language=bash, caption={Configurazione Git Remotes}]
# Remote per il repository personale (origin)
origin  https://github.com/mariocelzo/petclinic-dependability-analysis.git

# Remote per il repository originale Spring Projects (upstream)
upstream  https://github.com/spring-projects/spring-petclinic.git
\end{lstlisting}

Questa configurazione permette di:
\begin{itemize}
    \item Pushare modifiche e analisi sul repository personale (\texttt{origin})
    \item Sincronizzare aggiornamenti dal progetto originale (\texttt{upstream})
    \item Mantenere traccia delle divergenze rispetto alla baseline ufficiale
\end{itemize}

\subsubsection{Operazioni Eseguite}
\begin{lstlisting}[language=bash, caption={Comandi Git Setup}]
# 1. Rinomina remote originale in upstream
git remote rename origin upstream

# 2. Aggiunge nuovo remote per repository personale
git remote add origin https://github.com/mariocelzo/petclinic-dependability-analysis.git

# 3. Push iniziale del codice
git push -u origin main

# Output: 10,931 oggetti caricati (7.92 MiB)
\end{lstlisting}

\subsection{File di Configurazione Copiati}
\label{subsec:config-files}

Dalla directory \texttt{config/} del progetto principale sono stati copiati i seguenti file nella root del repository Spring PetClinic:

\subsubsection{sonar-project.properties}
File di configurazione per l'integrazione con SonarCloud.

\begin{lstlisting}[language=properties, caption={sonar-project.properties - Configurazione Principale}]
# Identificazione progetto
sonar.projectKey=mariocelzo_petclinic-dependability-analysis
sonar.organization=mariocelzo

# Informazioni progetto
sonar.projectName=Spring PetClinic Dependability Analysis
sonar.projectVersion=1.0

# Percorsi sorgenti e test
sonar.sources=src/main/java
sonar.tests=src/test/java

# Java version
sonar.java.source=21
sonar.java.binaries=target/classes

# Coverage report
sonar.coverage.jacoco.xmlReportPaths=target/site/jacoco/jacoco.xml

# Mutation testing
sonar.pitest.mode=reuseReport
sonar.pitest.reportsDirectory=target/pit-reports
\end{lstlisting}

\textbf{Dimensione}: 2.9 KB \\
\textbf{Scopo}: Configura SonarCloud per analizzare il codice Java, integrare report JaCoCo e PITest, definire percorsi sorgenti/test.

\subsubsection{checkstyle.xml}
Regole di code style basate su Google Java Style Guide con customizzazioni.

\begin{lstlisting}[language=xml, caption={checkstyle.xml - Regole Principali}]
<module name="Checker">
    <property name="severity" value="warning"/>
    
    <!-- Complessità ciclomatica massima -->
    <module name="CyclomaticComplexity">
        <property name="max" value="10"/>
    </module>
    
    <!-- Lunghezza massima metodi -->
    <module name="MethodLength">
        <property name="max" value="150"/>
    </module>
    
    <!-- Javadoc obbligatorio per API pubbliche -->
    <module name="JavadocMethod">
        <property name="scope" value="public"/>
    </module>
    
    <!-- Naming conventions -->
    <module name="TypeName"/>
    <module name="MethodName"/>
    <module name="ConstantName"/>
</module>
\end{lstlisting}

\textbf{Dimensione}: 6.6 KB \\
\textbf{Scopo}: Verifica conformità a standard di stile, complessità ciclomatica, lunghezza metodi, documentazione Javadoc.

\subsubsection{spotbugs-security.xml}
Filtri per SpotBugs e FindSecBugs per focus su security issues.

\begin{lstlisting}[language=xml, caption={spotbugs-security.xml - Filtri Security}]
<FindBugsFilter>
    <!-- Include tutti i security bug patterns -->
    <Match>
        <Bug category="SECURITY"/>
    </Match>
    
    <!-- Pattern specifici critici -->
    <Match>
        <Bug pattern="SQL_INJECTION_JDBC"/>
    </Match>
    <Match>
        <Bug pattern="PATH_TRAVERSAL_IN"/>
    </Match>
    <Match>
        <Bug pattern="XSS_REQUEST_PARAMETER_TO_SEND_ERROR"/>
    </Match>
    
    <!-- Esclusioni test -->
    <Match>
        <Class name="~.*\.Test.*"/>
        <Bug category="SECURITY"/>
    </Match>
</FindBugsFilter>
\end{lstlisting}

\textbf{Dimensione}: 4.8 KB \\
\textbf{Scopo}: Configura analisi statica security-focused, filtra false positive nei test.

\subsubsection{dependency-check-suppressions.xml}
File per sopprimere false positive nelle scansioni CVE di OWASP.

\begin{lstlisting}[language=xml, caption={dependency-check-suppressions.xml}]
<?xml version="1.0" encoding="UTF-8"?>
<suppressions xmlns="https://jeremylong.github.io/DependencyCheck/dependency-suppression.1.3.xsd">
    <!-- File vuoto inizialmente -->
    <!-- Verrà popolato con suppressioni validate durante l'analisi -->
</suppressions>
\end{lstlisting}

\textbf{Dimensione}: 685 bytes \\
\textbf{Scopo}: Gestione false positive CVE, da popolare durante analisi iterativa.

\subsection{GitHub Actions Workflows}
\label{subsec:github-actions}

Quattro workflow CI/CD sono stati copiati nella directory \texttt{.github/workflows/}:

\subsubsection{build.yml - Build e Test Base}
\begin{lstlisting}[language=yaml, caption={build.yml - Workflow Build}]
name: Build and Test
on: 
  push:
    branches: [main, develop]
  pull_request:
    branches: [main, develop]

jobs:
  build:
    runs-on: ubuntu-latest
    steps:
      - uses: actions/checkout@v4
      
      - name: Set up JDK 21
        uses: actions/setup-java@v4
        with:
          java-version: '21'
          distribution: 'temurin'
          cache: 'maven'
      
      - name: Build with Maven
        run: ./mvnw clean install -DskipTests
      
      - name: Run Tests
        run: ./mvnw test
      
      - name: Upload Test Results
        uses: actions/upload-artifact@v4
        if: always()
        with:
          name: test-results
          path: target/surefire-reports/
\end{lstlisting}

\textbf{Trigger}: Push su main/develop, Pull Request \\
\textbf{Durata stimata}: 3-5 minuti \\
\textbf{Output}: Test results come artifact

\subsubsection{quality.yml - Analisi Qualità}
\begin{lstlisting}[language=yaml, caption={quality.yml - Workflow Qualità (estratto)}]
name: Code Quality Analysis
on:
  push:
    branches: [main, develop]
  pull_request:
    branches: [main]
  schedule:
    - cron: '0 2 * * 1'  # Lunedì alle 2 AM

jobs:
  sonarcloud:
    name: SonarCloud Analysis
    runs-on: ubuntu-latest
    steps:
      - uses: actions/checkout@v4
        with:
          fetch-depth: 0  # Full history per SonarCloud
      
      - name: Set up JDK 21
        uses: actions/setup-java@v4
        with:
          java-version: '21'
          distribution: 'temurin'
      
      - name: Cache SonarCloud packages
        uses: actions/cache@v4
        with:
          path: ~/.sonar/cache
          key: ${{ runner.os }}-sonar
      
      - name: Run Tests with Coverage
        run: ./mvnw clean test jacoco:report
      
      - name: SonarCloud Scan
        env:
          GITHUB_TOKEN: ${{ secrets.GITHUB_TOKEN }}
          SONAR_TOKEN: ${{ secrets.SONAR_TOKEN }}
        run: |
          ./mvnw sonar:sonar \
            -Dsonar.projectKey=mariocelzo_petclinic-dependability-analysis \
            -Dsonar.organization=mariocelzo \
            -Dsonar.host.url=https://sonarcloud.io
      
      - name: Upload Coverage Report
        uses: actions/upload-artifact@v4
        with:
          name: coverage-report
          path: target/site/jacoco/

  mutation-testing:
    name: PITest Mutation Testing
    runs-on: ubuntu-latest
    steps:
      - uses: actions/checkout@v4
      
      - name: Set up JDK 21
        uses: actions/setup-java@v4
        with:
          java-version: '21'
          distribution: 'temurin'
      
      - name: Run Mutation Tests
        run: ./mvnw pitest:mutationCoverage
      
      - name: Upload Mutation Report
        uses: actions/upload-artifact@v4
        with:
          name: mutation-report
          path: target/pit-reports/
\end{lstlisting}

\textbf{Trigger}: Push, Pull Request, Schedule (weekly) \\
\textbf{Durata stimata}: 15-20 minuti \\
\textbf{Output}: Coverage report, Mutation report, SonarCloud dashboard \\
\textbf{Jobs}: SonarCloud analysis, Mutation testing (PITest)

\subsubsection{security.yml - Analisi Sicurezza}
\begin{lstlisting}[language=yaml, caption={security.yml - Workflow Security (estratto)}]
name: Security Analysis
on:
  push:
    branches: [main]
  schedule:
    - cron: '0 3 * * 0'  # Domenica alle 3 AM

jobs:
  dependency-check:
    name: OWASP Dependency Check
    runs-on: ubuntu-latest
    steps:
      - uses: actions/checkout@v4
      
      - name: Set up JDK 21
        uses: actions/setup-java@v4
        with:
          java-version: '21'
          distribution: 'temurin'
      
      - name: Run OWASP Dependency Check
        run: ./mvnw dependency-check:check
      
      - name: Upload Security Report
        uses: actions/upload-artifact@v4
        with:
          name: dependency-check-report
          path: target/dependency-check-report.html

  spotbugs:
    name: SpotBugs Security Analysis
    runs-on: ubuntu-latest
    steps:
      - uses: actions/checkout@v4
      
      - name: Set up JDK 21
        uses: actions/setup-java@v4
        with:
          java-version: '21'
          distribution: 'temurin'
      
      - name: Run SpotBugs
        run: ./mvnw spotbugs:check
      
      - name: Upload SpotBugs Report
        uses: actions/upload-artifact@v4
        with:
          name: spotbugs-report
          path: target/spotbugsXml.xml

  codeql:
    name: CodeQL Analysis
    runs-on: ubuntu-latest
    permissions:
      security-events: write
    steps:
      - uses: actions/checkout@v4
      
      - name: Initialize CodeQL
        uses: github/codeql-action/init@v3
        with:
          languages: java
      
      - name: Autobuild
        uses: github/codeql-action/autobuild@v3
      
      - name: Perform CodeQL Analysis
        uses: github/codeql-action/analyze@v3
\end{lstlisting}

\textbf{Trigger}: Push su main, Schedule (weekly) \\
\textbf{Durata stimata}: 10-15 minuti \\
\textbf{Output}: OWASP report, SpotBugs report, CodeQL security alerts \\
\textbf{Jobs}: OWASP Dependency-Check, SpotBugs+FindSecBugs, CodeQL

\subsubsection{docker.yml - Build Container}
\begin{lstlisting}[language=yaml, caption={docker.yml - Workflow Docker (estratto)}]
name: Docker Build and Push
on:
  push:
    tags:
      - 'v*'
  workflow_dispatch:

jobs:
  build:
    runs-on: ubuntu-latest
    steps:
      - uses: actions/checkout@v4
      
      - name: Set up JDK 21
        uses: actions/setup-java@v4
        with:
          java-version: '21'
          distribution: 'temurin'
      
      - name: Build with Maven
        run: ./mvnw clean package -DskipTests
      
      - name: Set up Docker Buildx
        uses: docker/setup-buildx-action@v3
      
      - name: Login to DockerHub
        uses: docker/login-action@v3
        with:
          username: ${{ secrets.DOCKER_USERNAME }}
          password: ${{ secrets.DOCKER_PASSWORD }}
      
      - name: Extract metadata
        id: meta
        uses: docker/metadata-action@v5
        with:
          images: mariocelzo/petclinic
          tags: |
            type=ref,event=tag
            type=semver,pattern={{version}}
            type=semver,pattern={{major}}.{{minor}}
            type=raw,value=latest
      
      - name: Build and push
        uses: docker/build-push-action@v5
        with:
          context: .
          push: true
          tags: ${{ steps.meta.outputs.tags }}
          labels: ${{ steps.meta.outputs.labels }}
\end{lstlisting}

\textbf{Trigger}: Push tag (v*), Manual workflow\_dispatch \\
\textbf{Durata stimata}: 5-8 minuti \\
\textbf{Output}: Docker image su DockerHub con tag versione e latest

\subsection{Modifica pom.xml: Compatibilità Java}
\label{subsec:pom-java-version}

Il file \texttt{pom.xml} richiedeva inizialmente Java 25, non disponibile. È stato modificato per utilizzare Java 21 LTS:

\begin{lstlisting}[language=xml, caption={pom.xml - Modifica versione Java}]
<properties>
    <!-- Prima: -->
    <!-- <java.version>25</java.version> -->
    <!-- <maven.compiler.release>17</maven.compiler.release> -->
    
    <!-- Dopo: -->
    <java.version>21</java.version>
    <maven.compiler.release>21</maven.compiler.release>
</properties>
\end{lstlisting}

\textbf{Motivazione}: Java 21 è la versione LTS disponibile, compatibile con tutte le dipendenze Spring Boot 4.0.0-M3.

\subsection{Aggiunta Plugin Maven}
\label{subsec:maven-plugins}

Nel file \texttt{pom.xml}, all'interno della sezione \texttt{<build><plugins>}, sono stati aggiunti i seguenti plugin:

\subsubsection{JaCoCo - Code Coverage}
\begin{lstlisting}[language=xml, caption={Plugin JaCoCo}]
<plugin>
    <groupId>org.jacoco</groupId>
    <artifactId>jacoco-maven-plugin</artifactId>
    <version>0.8.10</version>
    <executions>
        <execution>
            <id>prepare-agent</id>
            <goals>
                <goal>prepare-agent</goal>
            </goals>
        </execution>
        <execution>
            <id>report</id>
            <phase>test</phase>
            <goals>
                <goal>report</goal>
            </goals>
        </execution>
        <execution>
            <id>jacoco-check</id>
            <phase>verify</phase>
            <goals>
                <goal>check</goal>
            </goals>
            <configuration>
                <rules>
                    <rule>
                        <element>BUNDLE</element>
                        <limits>
                            <limit>
                                <counter>LINE</counter>
                                <value>COVEREDRATIO</value>
                                <minimum>0.80</minimum>
                            </limit>
                            <limit>
                                <counter>BRANCH</counter>
                                <value>COVEREDRATIO</value>
                                <minimum>0.75</minimum>
                            </limit>
                        </limits>
                    </rule>
                </rules>
            </configuration>
        </execution>
    </executions>
</plugin>
\end{lstlisting}

\textbf{Configurazione}:
\begin{itemize}
    \item \textbf{Line Coverage minima}: 80\%
    \item \textbf{Branch Coverage minima}: 75\%
    \item \textbf{Report format}: HTML + XML
    \item \textbf{Output}: \texttt{target/site/jacoco/}
\end{itemize}

\subsubsection{PITest - Mutation Testing}
\begin{lstlisting}[language=xml, caption={Plugin PITest}]
<plugin>
    <groupId>org.pitest</groupId>
    <artifactId>pitest-maven</artifactId>
    <version>1.14.2</version>
    <dependencies>
        <dependency>
            <groupId>org.pitest</groupId>
            <artifactId>pitest-junit5-plugin</artifactId>
            <version>1.2.0</version>
        </dependency>
    </dependencies>
    <configuration>
        <targetClasses>
            <param>org.springframework.samples.petclinic.*</param>
        </targetClasses>
        <targetTests>
            <param>org.springframework.samples.petclinic.*</param>
        </targetTests>
        <excludedClasses>
            <param>org.springframework.samples.petclinic.PetClinicApplication</param>
            <param>org.springframework.samples.petclinic.system.*</param>
        </excludedClasses>
        <mutators>
            <mutator>DEFAULTS</mutator>
        </mutators>
        <outputFormats>
            <outputFormat>XML</outputFormat>
            <outputFormat>HTML</outputFormat>
        </outputFormats>
        <timestampedReports>false</timestampedReports>
        <mutationThreshold>75</mutationThreshold>
        <coverageThreshold>80</coverageThreshold>
        <threads>4</threads>
        <timeoutConstant>3000</timeoutConstant>
        <verbose>false</verbose>
    </configuration>
</plugin>
\end{lstlisting}

\textbf{Configurazione}:
\begin{itemize}
    \item \textbf{Mutation Score minimo}: 75\%
    \item \textbf{Coverage threshold}: 80\%
    \item \textbf{Mutators}: DEFAULTS (tutti i mutatori standard)
    \item \textbf{Threads}: 4 (parallelizzazione)
    \item \textbf{Timeout}: 3000ms per mutazione
    \item \textbf{Output}: \texttt{target/pit-reports/}
\end{itemize}

\subsubsection{OWASP Dependency-Check}
\begin{lstlisting}[language=xml, caption={Plugin OWASP Dependency-Check}]
<plugin>
    <groupId>org.owasp</groupId>
    <artifactId>dependency-check-maven</artifactId>
    <version>8.4.0</version>
    <configuration>
        <format>ALL</format>
        <outputDirectory>${project.build.directory}</outputDirectory>
        <failBuildOnCVSS>7</failBuildOnCVSS>
        <suppressionFiles>
            <suppressionFile>dependency-check-suppressions.xml</suppressionFile>
        </suppressionFiles>
    </configuration>
    <executions>
        <execution>
            <goals>
                <goal>check</goal>
            </goals>
        </execution>
    </executions>
</plugin>
\end{lstlisting}

\textbf{Configurazione}:
\begin{itemize}
    \item \textbf{CVSS threshold}: 7.0 (HIGH) - build fallisce se trovate vulnerabilità >= 7.0
    \item \textbf{Format}: ALL (HTML, JSON, XML, CSV)
    \item \textbf{Suppressions}: File \texttt{dependency-check-suppressions.xml}
    \item \textbf{Output}: \texttt{target/dependency-check-report.html}
\end{itemize}

\subsubsection{SpotBugs + FindSecBugs}
\begin{lstlisting}[language=xml, caption={Plugin SpotBugs + FindSecBugs}]
<plugin>
    <groupId>com.github.spotbugs</groupId>
    <artifactId>spotbugs-maven-plugin</artifactId>
    <version>4.7.3.5</version>
    <configuration>
        <effort>Max</effort>
        <threshold>Low</threshold>
        <failOnError>false</failOnError>
        <includeFilterFile>spotbugs-security.xml</includeFilterFile>
        <plugins>
            <plugin>
                <groupId>com.h3xstream.findsecbugs</groupId>
                <artifactId>findsecbugs-plugin</artifactId>
                <version>1.12.0</version>
            </plugin>
        </plugins>
    </configuration>
    <executions>
        <execution>
            <goals>
                <goal>check</goal>
            </goals>
        </execution>
    </executions>
</plugin>
\end{lstlisting}

\textbf{Configurazione}:
\begin{itemize}
    \item \textbf{Effort}: Max (analisi più approfondita)
    \item \textbf{Threshold}: Low (report anche issue minori)
    \item \textbf{FindSecBugs}: Plugin security-focused per pattern vulnerabilità
    \item \textbf{Filter}: \texttt{spotbugs-security.xml} per focus su security
    \item \textbf{Output}: \texttt{target/spotbugsXml.xml}
\end{itemize}

\subsubsection{Checkstyle}
\begin{lstlisting}[language=xml, caption={Plugin Checkstyle}]
<plugin>
    <groupId>org.apache.maven.plugins</groupId>
    <artifactId>maven-checkstyle-plugin</artifactId>
    <version>3.3.0</version>
    <configuration>
        <configLocation>checkstyle.xml</configLocation>
        <encoding>UTF-8</encoding>
        <consoleOutput>true</consoleOutput>
        <failsOnError>false</failsOnError>
        <linkXRef>false</linkXRef>
    </configuration>
    <executions>
        <execution>
            <id>validate</id>
            <phase>validate</phase>
            <goals>
                <goal>check</goal>
            </goals>
        </execution>
    </executions>
</plugin>
\end{lstlisting}

\textbf{Configurazione}:
\begin{itemize}
    \item \textbf{Config}: \texttt{checkstyle.xml} custom rules
    \item \textbf{Phase}: validate (eseguito prima della compilazione)
    \item \textbf{Console output}: Abilitato per feedback immediato
    \item \textbf{Output}: \texttt{target/site/checkstyle.html}
\end{itemize}

\subsubsection{SonarQube Scanner}
\begin{lstlisting}[language=xml, caption={Plugin SonarQube Scanner}]
<plugin>
    <groupId>org.sonarsource.scanner.maven</groupId>
    <artifactId>sonar-maven-plugin</artifactId>
    <version>3.10.0.2594</version>
</plugin>
\end{lstlisting}

\textbf{Configurazione}: Utilizza \texttt{sonar-project.properties} per configurazione dettagliata.

\subsection{Riepilogo Modifiche pom.xml}
\label{subsec:pom-summary}

\begin{table}[h]
\centering
\begin{tabular}{|l|l|l|}
\hline
\textbf{Plugin} & \textbf{Versione} & \textbf{Scopo} \\
\hline
jacoco-maven-plugin & 0.8.10 & Code coverage analysis \\
pitest-maven & 1.14.2 & Mutation testing \\
dependency-check-maven & 8.4.0 & CVE vulnerability scanning \\
spotbugs-maven-plugin & 4.7.3.5 & Static analysis + security \\
maven-checkstyle-plugin & 3.3.0 & Code style verification \\
sonar-maven-plugin & 3.10.0.2594 & SonarCloud integration \\
\hline
\end{tabular}
\caption{Plugin Maven aggiunti per analisi dependability}
\label{tab:maven-plugins-added}
\end{table}

\textbf{Totale righe aggiunte}: 178 linee XML \\
\textbf{Dimensione pom.xml finale}: 628 righe (da 450)

\subsection{Commit e Push su GitHub}
\label{subsec:git-commits}

Due commit sono stati creati e pushati sul repository GitHub:

\subsubsection{Commit 1: Configurazione File}
\begin{lstlisting}[language=bash, caption={Commit 1 - File di Configurazione}]
Commit: 39fc5d2
Author: Mario Celzo
Date: 21 Novembre 2025
Message: feat: Add dependability analysis configuration

Modifiche:
- Add SonarCloud configuration (sonar-project.properties)
- Add Checkstyle rules (checkstyle.xml)
- Add SpotBugs security filters (spotbugs-security.xml)
- Add OWASP Dependency-Check suppressions
- Add GitHub Actions workflows (build, quality, security, docker)
- Update Java version to 21 in pom.xml for compatibility

Files changed: 9 files, 860 insertions(+), 2 deletions(-)
\end{lstlisting}

\subsubsection{Commit 2: Plugin Maven}
\begin{lstlisting}[language=bash, caption={Commit 2 - Plugin Maven}]
Commit: 27ac291
Author: Mario Celzo
Date: 21 Novembre 2025
Message: feat: Add Maven plugins for dependability analysis

Modifiche:
- Add JaCoCo plugin for code coverage analysis (0.8.10)
- Add PITest plugin for mutation testing (1.14.2)
- Add OWASP Dependency-Check for security vulnerability scanning (8.4.0)
- Add SpotBugs + FindSecBugs for static security analysis (4.7.3.5)
- Add Checkstyle plugin for code style verification (3.3.0)
- Add SonarQube scanner plugin for cloud-based quality analysis

Configured thresholds:
- Line coverage: 80%
- Branch coverage: 75%
- Mutation score: 75%
- CVSS fail threshold: 7.0

Files changed: 1 file, 178 insertions(+)
\end{lstlisting}

\subsection{Stato Finale Repository}
\label{subsec:final-state}

\subsubsection{File Aggiunti alla Root}
\begin{itemize}
    \item \texttt{sonar-project.properties} (2.9 KB)
    \item \texttt{checkstyle.xml} (6.6 KB)
    \item \texttt{spotbugs-security.xml} (4.8 KB)
    \item \texttt{dependency-check-suppressions.xml} (685 bytes)
\end{itemize}

\subsubsection{Workflow GitHub Actions}
\begin{itemize}
    \item \texttt{.github/workflows/build.yml} (1.7 KB)
    \item \texttt{.github/workflows/quality.yml} (3.0 KB)
    \item \texttt{.github/workflows/security.yml} (3.4 KB)
    \item \texttt{.github/workflows/docker.yml} (2.3 KB)
\end{itemize}

\subsubsection{Modifiche File Esistenti}
\begin{itemize}
    \item \texttt{pom.xml}: +178 righe (plugin analisi)
    \item Java version: 25 → 21
    \item Compiler release: 17 → 21
\end{itemize}

\subsection{Verifica Configurazione}
\label{subsec:configuration-verification}

Per verificare la corretta installazione dei plugin, eseguire:

\begin{lstlisting}[language=bash, caption={Verifica Plugin Installati}]
# Verifica plugin disponibili
./mvnw help:describe -Dplugin=org.jacoco:jacoco-maven-plugin
./mvnw help:describe -Dplugin=org.pitest:pitest-maven
./mvnw help:describe -Dplugin=org.owasp:dependency-check-maven

# Verifica obiettivi (goals) disponibili
./mvnw help:describe -Dplugin=jacoco -Ddetail

# Lista tutti i plugin del progetto
./mvnw help:effective-pom | grep -A 2 "<artifactId>.*-plugin</artifactId>"
\end{lstlisting}

\subsection{Prossimi Passi Configurazione}
\label{subsec:next-config-steps}

\subsubsection{Setup SonarCloud}
\begin{enumerate}
    \item Registrazione su \url{https://sonarcloud.io} con account GitHub
    \item Import repository \texttt{mariocelzo/petclinic-dependability-analysis}
    \item Generazione \texttt{SONAR\_TOKEN}
    \item Aggiunta secret in GitHub: Settings → Secrets and variables → Actions → New repository secret
    \item Nome secret: \texttt{SONAR\_TOKEN}
\end{enumerate}

\subsubsection{Setup DockerHub (Opzionale)}
\begin{enumerate}
    \item Creazione account DockerHub o login esistente
    \item Creazione repository: \texttt{mariocelzo/petclinic}
    \item Generazione Access Token
    \item Aggiunta secrets in GitHub:
    \begin{itemize}
        \item \texttt{DOCKER\_USERNAME}: username DockerHub
        \item \texttt{DOCKER\_PASSWORD}: access token generato
    \end{itemize}
\end{enumerate}

\subsubsection{Prima Esecuzione Analisi}
\begin{lstlisting}[language=bash, caption={Prima Esecuzione Locale}]
# 1. Coverage baseline
./mvnw clean test jacoco:report
# Apri: target/site/jacoco/index.html

# 2. Checkstyle (veloce)
./mvnw checkstyle:check
# Apri: target/site/checkstyle.html

# 3. SpotBugs (medio, ~2-3 min)
./mvnw spotbugs:check
# Apri: target/spotbugsXml.xml

# 4. OWASP Dependency-Check (lento, ~5-10 min prima volta)
./mvnw dependency-check:check
# Apri: target/dependency-check-report.html

# 5. Mutation Testing (molto lento, ~10-15 min)
./mvnw pitest:mutationCoverage
# Apri: target/pit-reports/index.html
\end{lstlisting}

\subsection{Integrazione Documentazione LaTeX}
\label{subsec:latex-documentation}

È stato creato il file \texttt{docs/report/sections/08-analysis-tools.tex} con documentazione dettagliata su:
\begin{itemize}
    \item Architettura directory \texttt{analysis/}
    \item Descrizione di ogni strumento (SonarCloud, JaCoCo, PITest, OWASP, SpotBugs, Checkstyle)
    \item File di configurazione e loro scopo
    \item Metriche monitorate e soglie
    \item Workflow di analisi completo
    \item Best practices e raccomandazioni
\end{itemize}

Questo file può essere importato nel documento principale LaTeX con:
\begin{lstlisting}[language=tex]
\section{Strumenti di Analisi e Configurazione}
\label{sec:analysis-tools}

Questa sezione descrive in dettaglio gli strumenti di analisi utilizzati nel progetto, la loro configurazione e la struttura organizzativa della directory \texttt{analysis/}.

\subsection{Architettura della Directory di Analisi}
\label{subsec:analysis-structure}

La directory \texttt{analysis/} è organizzata in sei sottocartelle tematiche, ognuna dedicata a un aspetto specifico dell'analisi di dependability:

\begin{lstlisting}[language=bash, caption={Struttura directory analysis/}]
analysis/
├── 01-sonarcloud/          # Analisi qualità codice
├── 02-coverage/            # Analisi copertura test
├── 03-mutation/            # Mutation testing
├── 04-performance/         # Analisi performance
├── 05-security/            # Analisi sicurezza
└── 06-test-generation/     # Generazione test automatici
\end{lstlisting}

\subsection{01 - SonarCloud: Analisi Qualità del Codice}
\label{subsec:sonarcloud}

\subsubsection{Descrizione dello Strumento}
SonarCloud è una piattaforma di analisi statica del codice cloud-based che identifica:
\begin{itemize}
    \item \textbf{Bug}: Errori nel codice che possono causare comportamenti inattesi
    \item \textbf{Vulnerabilità}: Problemi di sicurezza che espongono il sistema ad attacchi
    \item \textbf{Code Smells}: Violazioni delle best practices che riducono la manutenibilità
    \item \textbf{Duplicazioni}: Codice duplicato che aumenta il debito tecnico
    \item \textbf{Coverage Gap}: Aree del codice non coperte da test
\end{itemize}

\subsubsection{File di Configurazione}
Il file \texttt{sonar-project.properties} nella root del progetto contiene:

\begin{lstlisting}[language=properties, caption={sonar-project.properties}]
# Identificazione progetto
sonar.projectKey=mariocelzo_petclinic-dependability-analysis
sonar.organization=mariocelzo

# Informazioni progetto
sonar.projectName=Spring PetClinic Dependability Analysis
sonar.projectVersion=1.0

# Percorsi sorgenti e test
sonar.sources=src/main/java
sonar.tests=src/test/java

# Esclusioni
sonar.exclusions=**/target/**,**/node_modules/**

# Java version
sonar.java.source=21
sonar.java.binaries=target/classes

# Coverage report
sonar.coverage.jacoco.xmlReportPaths=target/site/jacoco/jacoco.xml

# Mutation testing (PITest)
sonar.pitest.mode=reuseReport
sonar.pitest.reportsDirectory=target/pit-reports
\end{lstlisting}

\subsubsection{Metriche Chiave Monitorate}
\begin{table}[h]
\centering
\begin{tabular}{|l|l|p{6cm}|}
\hline
\textbf{Metrica} & \textbf{Soglia} & \textbf{Descrizione} \\
\hline
Reliability Rating & A & Numero di bug per 1000 righe di codice \\
Security Rating & A & Numero di vulnerabilità critiche \\
Maintainability Rating & A & Rapporto debito tecnico/tempo sviluppo \\
Coverage & \textgreater 80\% & Percentuale codice coperto da test \\
Duplications & \textless 3\% & Percentuale codice duplicato \\
\hline
\end{tabular}
\caption{Metriche SonarCloud e soglie di qualità}
\label{tab:sonarcloud-metrics}
\end{table}

\subsubsection{Contenuto Directory}
\begin{itemize}
    \item \texttt{README.md}: Guida all'uso di SonarCloud e interpretazione metriche
    \item \texttt{issues-report.md}: Report dettagliato delle issue trovate (bug, vulnerabilità, code smells)
    \item \texttt{screenshots/}: Screenshot dell'interfaccia SonarCloud con dashboard metriche
\end{itemize}

\subsection{02 - Coverage: Analisi Copertura Test}
\label{subsec:coverage}

\subsubsection{Descrizione dello Strumento}
JaCoCo (Java Code Coverage) è uno strumento di analisi della copertura del codice che misura:
\begin{itemize}
    \item \textbf{Line Coverage}: Percentuale di righe eseguite dai test
    \item \textbf{Branch Coverage}: Percentuale di branch decisionali testati
    \item \textbf{Method Coverage}: Percentuale di metodi invocati
    \item \textbf{Class Coverage}: Percentuale di classi istanziate
    \item \textbf{Instruction Coverage}: Percentuale di bytecode eseguito
\end{itemize}

\subsubsection{Configurazione Maven Plugin}
Nel file \texttt{pom.xml}, il plugin JaCoCo è configurato come segue:

\begin{lstlisting}[language=xml, caption={Configurazione JaCoCo in pom.xml}]
<plugin>
    <groupId>org.jacoco</groupId>
    <artifactId>jacoco-maven-plugin</artifactId>
    <version>0.8.10</version>
    <executions>
        <execution>
            <goals>
                <goal>prepare-agent</goal>
            </goals>
        </execution>
        <execution>
            <id>report</id>
            <phase>test</phase>
            <goals>
                <goal>report</goal>
            </goals>
        </execution>
        <execution>
            <id>jacoco-check</id>
            <goals>
                <goal>check</goal>
            </goals>
            <configuration>
                <rules>
                    <rule>
                        <element>PACKAGE</element>
                        <limits>
                            <limit>
                                <counter>LINE</counter>
                                <value>COVEREDRATIO</value>
                                <minimum>0.80</minimum>
                            </limit>
                        </limits>
                    </rule>
                </rules>
            </configuration>
        </execution>
    </executions>
</plugin>
\end{lstlisting}

\subsubsection{Comando di Esecuzione}
\begin{lstlisting}[language=bash]
# Esegue test e genera report coverage
./mvnw clean test jacoco:report

# Il report HTML è generato in:
# target/site/jacoco/index.html
\end{lstlisting}

\subsubsection{Contenuto Directory}
\begin{itemize}
    \item \texttt{README.md}: Documentazione uso JaCoCo e interpretazione metriche
    \item \texttt{coverage-analysis.md}: Analisi dettagliata della copertura per package, identificazione gap critici
    \item \texttt{graphs/}: Grafici evoluzione copertura nel tempo (line chart, heatmap per package)
\end{itemize}

\subsection{03 - Mutation: Mutation Testing}
\label{subsec:mutation}

\subsubsection{Descrizione dello Strumento}
PITest (PIT Mutation Testing) è uno strumento che valuta la qualità dei test introducendo mutazioni nel codice:

\begin{enumerate}
    \item \textbf{Mutazione}: Modifica intenzionale del codice sorgente (es: \texttt{>} diventa \texttt{>=})
    \item \textbf{Esecuzione Test}: I test vengono eseguiti contro il codice mutato
    \item \textbf{Risultato}:
    \begin{itemize}
        \item \textbf{KILLED}: Test fallisce → Test efficace (buono)
        \item \textbf{SURVIVED}: Test passa → Test debole (da migliorare)
        \item \textbf{NO\_COVERAGE}: Mutazione in codice non coperto
    \end{itemize}
\end{enumerate}

\subsubsection{Tipi di Mutazioni Applicate}
\begin{table}[h]
\centering
\begin{tabular}{|l|l|l|}
\hline
\textbf{Mutatore} & \textbf{Mutazione} & \textbf{Esempio} \\
\hline
CONDITIONALS\_BOUNDARY & Modifica condizioni & \texttt{<} → \texttt{<=} \\
INCREMENTS & Inverte incrementi & \texttt{i++} → \texttt{i--} \\
INVERT\_NEGS & Inverte negazioni & \texttt{-x} → \texttt{+x} \\
MATH & Modifica operatori & \texttt{+} → \texttt{-} \\
NEGATE\_CONDITIONALS & Nega condizioni & \texttt{==} → \texttt{!=} \\
VOID\_METHOD\_CALLS & Rimuove chiamate void & \texttt{method()} → \texttt{//} \\
RETURN\_VALS & Modifica return & \texttt{return true} → \texttt{return false} \\
\hline
\end{tabular}
\caption{Principali mutatori PITest}
\label{tab:pitest-mutators}
\end{table}

\subsubsection{Configurazione Maven Plugin}
\begin{lstlisting}[language=xml, caption={Configurazione PITest in pom.xml}]
<plugin>
    <groupId>org.pitest</groupId>
    <artifactId>pitest-maven</artifactId>
    <version>1.14.2</version>
    <dependencies>
        <dependency>
            <groupId>org.pitest</groupId>
            <artifactId>pitest-junit5-plugin</artifactId>
            <version>1.2.0</version>
        </dependency>
    </dependencies>
    <configuration>
        <targetClasses>
            <param>org.springframework.samples.petclinic.*</param>
        </targetClasses>
        <targetTests>
            <param>org.springframework.samples.petclinic.*</param>
        </targetTests>
        <outputFormats>
            <outputFormat>HTML</outputFormat>
            <outputFormat>XML</outputFormat>
        </outputFormats>
        <mutationThreshold>70</mutationThreshold>
        <coverageThreshold>80</coverageThreshold>
        <threads>4</threads>
        <timestampedReports>false</timestampedReports>
    </configuration>
</plugin>
\end{lstlisting}

\subsubsection{Comando di Esecuzione}
\begin{lstlisting}[language=bash]
# Esegue mutation testing (10-15 minuti)
./mvnw pitest:mutationCoverage

# Report HTML generato in:
# target/pit-reports/index.html
\end{lstlisting}

\subsubsection{Metriche Mutation Score}
Il \textbf{Mutation Score} è calcolato come:
\[
\text{Mutation Score} = \frac{\text{Mutazioni KILLED}}{\text{Mutazioni Totali - NO\_COVERAGE}} \times 100
\]

\begin{itemize}
    \item \textbf{\textgreater 80\%}: Eccellente - Test molto robusti
    \item \textbf{60-80\%}: Buono - Test adeguati con margini di miglioramento
    \item \textbf{40-60\%}: Sufficiente - Test base ma con lacune significative
    \item \textbf{\textless 40\%}: Insufficiente - Test deboli, necessari miglioramenti urgenti
\end{itemize}

\subsubsection{Contenuto Directory}
\begin{itemize}
    \item \texttt{README.md}: Guida mutation testing e interpretazione risultati
    \item \texttt{mutation-analysis.md}: Analisi dettagliata mutation score per package e classe
    \item \texttt{weak-spots.md}: Identificazione punti deboli con mutazioni sopravvissute e raccomandazioni per test migliorati
    \item \texttt{reports/}: Report HTML/XML PITest per reference
\end{itemize}

\subsection{04 - Performance: Analisi Performance}
\label{subsec:performance}

\subsubsection{Descrizione dello Strumento}
L'analisi delle performance utilizza JMeter per test di carico e profiling con VisualVM/JProfiler:

\begin{itemize}
    \item \textbf{JMeter}: Test di carico HTTP con simulazione utenti concorrenti
    \item \textbf{VisualVM}: Profiling CPU, memoria, thread
    \item \textbf{Spring Boot Actuator}: Metriche runtime applicazione
\end{itemize}

\subsubsection{Scenari di Test}
\begin{enumerate}
    \item \textbf{Baseline}: 10 utenti, 100 richieste, measure baseline response time
    \item \textbf{Load Test}: 50 utenti, 500 richieste, verifica performance sotto carico normale
    \item \textbf{Stress Test}: 100+ utenti, 1000+ richieste, identifica breaking point
    \item \textbf{Endurance Test}: 20 utenti, 2 ore, verifica memory leaks
\end{enumerate}

\subsubsection{Metriche Monitorate}
\begin{table}[h]
\centering
\begin{tabular}{|l|l|l|}
\hline
\textbf{Metrica} & \textbf{Soglia Accettabile} & \textbf{Descrizione} \\
\hline
Response Time (avg) & \textless 200ms & Tempo medio risposta \\
Response Time (p95) & \textless 500ms & 95° percentile tempo risposta \\
Response Time (p99) & \textless 1000ms & 99° percentile tempo risposta \\
Throughput & \textgreater 100 req/s & Richieste elaborate per secondo \\
Error Rate & \textless 1\% & Percentuale richieste fallite \\
CPU Usage & \textless 70\% & Utilizzo CPU medio \\
Memory Usage & \textless 80\% & Utilizzo memoria heap \\
\hline
\end{tabular}
\caption{Metriche performance e soglie}
\label{tab:performance-metrics}
\end{table}

\subsubsection{Contenuto Directory}
\begin{itemize}
    \item \texttt{README.md}: Guida setup JMeter e esecuzione test performance
    \item \texttt{performance-analysis.md}: Analisi dettagliata risultati per scenario
    \item \texttt{bottlenecks.md}: Identificazione colli di bottiglia (query lente, N+1 problems, memory leaks) e ottimizzazioni proposte
    \item \texttt{results/}: File JTL JMeter, grafici response time, CPU/memory profiling
\end{itemize}

\subsection{05 - Security: Analisi Sicurezza}
\label{subsec:security}

\subsubsection{Strumenti di Analisi}
L'analisi di sicurezza utilizza un approccio multi-layer con quattro strumenti complementari:

\paragraph{OWASP Dependency-Check}
Scansiona le dipendenze Maven per vulnerabilità note (CVE):
\begin{lstlisting}[language=bash]
./mvnw dependency-check:check
# Report: target/dependency-check-report.html
\end{lstlisting}

\paragraph{SpotBugs + FindSecBugs}
Analisi statica per pattern di sicurezza problematici:
\begin{itemize}
    \item SQL Injection
    \item XSS (Cross-Site Scripting)
    \item Path Traversal
    \item Weak Cryptography
    \item Insecure Random
\end{itemize}

Configurazione in \texttt{spotbugs-security.xml}:
\begin{lstlisting}[language=xml, caption={spotbugs-security.xml - Filtri Security}]
<FindBugsFilter>
    <!-- Include tutti i security bug patterns -->
    <Match>
        <Bug category="SECURITY"/>
    </Match>
    
    <!-- Esclusioni per false positive -->
    <Match>
        <Class name="~.*\.Test.*"/>
        <Bug pattern="SQL_INJECTION_JDBC"/>
    </Match>
</FindBugsFilter>
\end{lstlisting}

\paragraph{GitHub CodeQL}
Analisi semantica del codice per identificare:
\begin{itemize}
    \item Command Injection
    \item LDAP Injection
    \item Unsafe Deserialization
    \item Hardcoded Credentials
\end{itemize}

\paragraph{Trivy}
Scansione vulnerabilità container Docker:
\begin{lstlisting}[language=bash]
trivy image petclinic:latest
\end{lstlisting}

\subsubsection{Configurazione File dependency-check-suppressions.xml}
File utilizzato per sopprimere false positive:
\begin{lstlisting}[language=xml, caption={dependency-check-suppressions.xml}]
<?xml version="1.0" encoding="UTF-8"?>
<suppressions xmlns="https://jeremylong.github.io/DependencyCheck/dependency-suppression.1.3.xsd">
    <!-- Esempio: Soppressione CVE già fixata in versione specifica -->
    <suppress>
        <notes>CVE fixata in Spring Boot 4.0.0-M3</notes>
        <cve>CVE-2023-XXXXX</cve>
    </suppress>
</suppressions>
\end{lstlisting}

\subsubsection{Contenuto Directory}
\begin{itemize}
    \item \texttt{README.md}: Guida esecuzione analisi sicurezza multi-tool
    \item \texttt{security-assessment.md}: Report completo vulnerabilità trovate categorizzate per OWASP Top 10, severity rating (Critical/High/Medium/Low), remediation plan
    \item \texttt{reports/}: Report HTML/JSON dei vari tool (OWASP, SpotBugs, CodeQL, Trivy)
\end{itemize}

\subsection{06 - Test Generation: Generazione Test Automatici}
\label{subsec:test-generation}

\subsubsection{Descrizione dello Strumento}
Utilizzo di strumenti di generazione automatica test per migliorare coverage:

\begin{itemize}
    \item \textbf{EvoSuite}: Generazione test basata su algoritmi genetici
    \item \textbf{Randoop}: Generazione test randomici con feedback
    \item \textbf{GitHub Copilot}: Suggerimenti test assistiti da AI
\end{itemize}

\subsubsection{EvoSuite - Approccio Genetico}
EvoSuite usa algoritmi evolutivi per generare test che massimizzano coverage:

\begin{lstlisting}[language=bash]
# Genera test per una classe specifica
java -jar evosuite.jar \
  -class org.springframework.samples.petclinic.owner.OwnerController \
  -projectCP target/classes \
  -Dsearch_budget=120
\end{lstlisting}

Parametri chiave:
\begin{itemize}
    \item \texttt{search\_budget}: Tempo in secondi per generazione (default: 60)
    \item \texttt{criterion}: Criterio di coverage (BRANCH, LINE, MUTATION)
    \item \texttt{test\_dir}: Directory output test generati
\end{itemize}

\subsubsection{Processo di Validazione Test Generati}
\begin{enumerate}
    \item \textbf{Generazione}: Tool produce test automatici
    \item \textbf{Review}: Analisi manuale per validità semantica
    \item \textbf{Refactoring}: Pulizia e miglioramento leggibilità
    \item \textbf{Integration}: Integrazione nella suite di test esistente
    \item \textbf{Maintenance}: Verifica test non diventino flaky
\end{enumerate}

\subsubsection{Metriche di Valutazione}
\begin{table}[h]
\centering
\begin{tabular}{|l|l|}
\hline
\textbf{Metrica} & \textbf{Descrizione} \\
\hline
Coverage Improvement & Delta coverage prima/dopo generazione \\
Test Validity Rate & \% test generati semanticamente corretti \\
Assertion Quality & Numero asserzioni significative per test \\
Flakiness Rate & \% test instabili (fail intermittenti) \\
Maintenance Cost & Effort necessario per manutenzione test \\
\hline
\end{tabular}
\caption{Metriche valutazione test generati}
\label{tab:test-generation-metrics}
\end{table}

\subsubsection{Contenuto Directory}
\begin{itemize}
    \item \texttt{README.md}: Guida uso EvoSuite/Randoop e best practices
    \item \texttt{generation-report.md}: Report dettagliato test generati per classe, coverage improvement, analisi qualità asserzioni, identificazione test da mantenere vs scartare
    \item \texttt{generated-tests/}: Test generati organizzati per package, prima della review/refactoring
\end{itemize}

\subsection{Workflow di Analisi Completo}
\label{subsec:analysis-workflow}

Il workflow completo di analisi segue questo ordine:

\begin{enumerate}
    \item \textbf{Setup Iniziale}: Copia file configurazione, aggiunta plugin Maven
    \item \textbf{Baseline Coverage}: Esegui \texttt{./mvnw test jacoco:report} per baseline
    \item \textbf{Mutation Testing}: Esegui \texttt{./mvnw pitest:mutationCoverage} (10-15 min)
    \item \textbf{Static Analysis}: Push su GitHub → trigger workflow quality.yml con SonarCloud
    \item \textbf{Security Scan}: Workflow security.yml esegue OWASP + SpotBugs + CodeQL
    \item \textbf{Performance Test}: Esegui test JMeter, raccogli metriche
    \item \textbf{Test Generation}: Genera test con EvoSuite per classi con bassa coverage
    \item \textbf{Iterazione}: Re-run coverage e mutation per verificare miglioramenti
    \item \textbf{Documentation}: Popola directory analysis/ con report e analisi
\end{enumerate}

\begin{figure}[h]
\centering
\begin{tikzpicture}[node distance=1.5cm, auto]
    % Definizione stili
    \tikzstyle{process} = [rectangle, draw, fill=blue!20, text width=3cm, text centered, rounded corners, minimum height=1cm]
    \tikzstyle{decision} = [diamond, draw, fill=green!20, text width=2cm, text centered, minimum height=1cm]
    \tikzstyle{arrow} = [thick,->,>=stealth]
    
    % Nodi
    \node [process] (setup) {Setup Iniziale};
    \node [process, below of=setup] (coverage) {Baseline Coverage};
    \node [process, below of=coverage] (mutation) {Mutation Testing};
    \node [process, below of=mutation] (static) {Static Analysis};
    \node [process, below of=static] (security) {Security Scan};
    \node [decision, below of=security] (check) {Issues Found?};
    \node [process, left of=check, xshift=-3cm] (fix) {Fix Issues};
    \node [process, below of=check, yshift=-1cm] (performance) {Performance Test};
    \node [process, below of=performance] (generation) {Test Generation};
    \node [process, below of=generation] (document) {Documentation};
    
    % Collegamenti
    \draw [arrow] (setup) -- (coverage);
    \draw [arrow] (coverage) -- (mutation);
    \draw [arrow] (mutation) -- (static);
    \draw [arrow] (static) -- (security);
    \draw [arrow] (security) -- (check);
    \draw [arrow] (check) -- node {Yes} (fix);
    \draw [arrow] (fix) |- (coverage);
    \draw [arrow] (check) -- node {No} (performance);
    \draw [arrow] (performance) -- (generation);
    \draw [arrow] (generation) -- (document);
\end{tikzpicture}
\caption{Workflow di analisi dependability}
\label{fig:analysis-workflow}
\end{figure}

\subsection{Integrazione con CI/CD}
\label{subsec:cicd-integration}

I workflow GitHub Actions nella directory \texttt{.github/workflows/} automatizzano le analisi:

\subsubsection{build.yml - Build e Test Base}
\begin{lstlisting}[language=yaml, caption={.github/workflows/build.yml}]
name: Build and Test
on: [push, pull_request]

jobs:
  build:
    runs-on: ubuntu-latest
    steps:
      - uses: actions/checkout@v4
      - uses: actions/setup-java@v4
        with:
          java-version: '21'
          distribution: 'temurin'
      - name: Build with Maven
        run: ./mvnw clean install
      - name: Upload test results
        uses: actions/upload-artifact@v4
        with:
          name: test-results
          path: target/surefire-reports/
\end{lstlisting}

\subsubsection{quality.yml - Analisi Qualità}
Esegue SonarCloud, JaCoCo coverage, PITest mutation testing:
\begin{itemize}
    \item Trigger: Push su main/develop, Pull Request
    \item Durata: ~15-20 minuti
    \item Artifacts: Coverage report, mutation report, SonarCloud dashboard
\end{itemize}

\subsubsection{security.yml - Analisi Sicurezza}
Esegue OWASP Dependency-Check, SpotBugs+FindSecBugs, CodeQL, Trivy:
\begin{itemize}
    \item Trigger: Push su main, schedule (weekly)
    \item Durata: ~10-15 minuti
    \item Artifacts: Security reports, vulnerability list
    \item Blocca build se vulnerabilità Critical/High
\end{itemize}

\subsubsection{docker.yml - Build e Push Container}
Build immagine Docker, scan con Trivy, push su DockerHub:
\begin{itemize}
    \item Trigger: Push tag (es: v1.0.0), manual workflow\_dispatch
    \item Durata: ~5-8 minuti
    \item Output: Immagine su DockerHub con tag versione e latest
\end{itemize}

\subsection{Best Practices e Raccomandazioni}
\label{subsec:best-practices}

\subsubsection{Frequenza Esecuzione Analisi}
\begin{itemize}
    \item \textbf{Coverage}: Ad ogni commit (veloce, 2-3 minuti)
    \item \textbf{Mutation}: Daily/prima di merge PR (lento, 10-15 minuti)
    \item \textbf{Static Analysis}: Ad ogni push (integrato in CI)
    \item \textbf{Security Scan}: Weekly + prima di release
    \item \textbf{Performance Test}: Prima di release + quando cambiano query/logica
\end{itemize}

\subsubsection{Gestione False Positive}
\begin{enumerate}
    \item \textbf{Documenta}: Spiega perché un issue è false positive
    \item \textbf{Sopprimi}: Usa file suppressions (dependency-check, spotbugs)
    \item \textbf{Configura}: Ajusta thresholds se troppo stringenti
    \item \textbf{Review}: Periodicamente re-valuta suppressions (potrebbero diventare veri issue)
\end{enumerate}

\subsubsection{Interpretazione Risultati}
\begin{itemize}
    \item \textbf{Non inseguire 100\%}: Coverage 80-85\% è ottimale (diminishing returns dopo)
    \item \textbf{Priorità}: Focus su codice critico (business logic, security, persistenza)
    \item \textbf{Trend}: Monitora evoluzione metriche nel tempo, non solo valore assoluto
    \item \textbf{Contestualizza}: Confronta con benchmark progetti simili
\end{itemize}

\subsection{Checkstyle: Analisi Code Style}
\label{subsec:checkstyle}

\subsubsection{Descrizione dello Strumento}
Checkstyle verifica conformità codice a standard di stile configurabili:

\begin{itemize}
    \item \textbf{Naming Conventions}: Nomi variabili, metodi, classi
    \item \textbf{Formatting}: Indentazione, spazi, parentesi
    \item \textbf{Javadoc}: Completezza documentazione
    \item \textbf{Design}: Numero parametri, complessità ciclomatica
    \item \textbf{Imports}: Ordine e wildcard imports
\end{itemize}

\subsubsection{File di Configurazione}
Il file \texttt{checkstyle.xml} nella root definisce le regole:

\begin{lstlisting}[language=xml, caption={checkstyle.xml - Regole principali}]
<?xml version="1.0"?>
<!DOCTYPE module PUBLIC
  "-//Checkstyle//DTD Checkstyle Configuration 1.3//EN"
  "https://checkstyle.org/dtds/configuration_1_3.dtd">

<module name="Checker">
    <property name="severity" value="warning"/>
    
    <!-- File formatting -->
    <module name="FileTabCharacter"/>
    <module name="NewlineAtEndOfFile"/>
    
    <module name="TreeWalker">
        <!-- Naming conventions -->
        <module name="TypeName"/>
        <module name="MethodName"/>
        <module name="ConstantName"/>
        <module name="LocalVariableName"/>
        
        <!-- Javadoc -->
        <module name="JavadocMethod">
            <property name="scope" value="public"/>
        </module>
        <module name="JavadocType"/>
        
        <!-- Code complexity -->
        <module name="CyclomaticComplexity">
            <property name="max" value="10"/>
        </module>
        <module name="MethodLength">
            <property name="max" value="150"/>
        </module>
        
        <!-- Best practices -->
        <module name="EmptyBlock"/>
        <module name="NeedBraces"/>
        <module name="LeftCurly"/>
        <module name="RightCurly"/>
    </module>
</module>
\end{lstlisting}

\subsubsection{Esecuzione}
\begin{lstlisting}[language=bash]
# Verifica code style
./mvnw checkstyle:check

# Genera report HTML
./mvnw checkstyle:checkstyle
# Report: target/site/checkstyle.html
\end{lstlisting}

Le violazioni Checkstyle sono integrate nel workflow \texttt{quality.yml} e reportate su SonarCloud come "Code Smells".

\subsection{Riepilogo File di Configurazione}
\label{subsec:config-summary}

\begin{table}[h]
\centering
\small
\begin{tabular}{|l|l|p{5cm}|}
\hline
\textbf{File} & \textbf{Tool} & \textbf{Scopo} \\
\hline
sonar-project.properties & SonarCloud & Configurazione progetto, paths, coverage \\
checkstyle.xml & Checkstyle & Regole code style e formatting \\
spotbugs-security.xml & SpotBugs & Filtri security patterns \\
dependency-check-suppressions.xml & OWASP & Suppressione false positive CVE \\
pom.xml & Maven & Plugin JaCoCo, PITest, OWASP, SpotBugs \\
.github/workflows/*.yml & GitHub Actions & CI/CD pipeline automatizzata \\
\hline
\end{tabular}
\caption{Riepilogo file di configurazione analisi}
\label{tab:config-files}
\end{table}

\subsection{Prossimi Passi}
\label{subsec:next-steps}

Dopo il setup della configurazione:

\begin{enumerate}
    \item Commit e push dei file di configurazione su GitHub
    \item Esecuzione prima analisi coverage: \texttt{./mvnw clean test jacoco:report}
    \item Setup account SonarCloud e configurazione secrets GitHub
    \item Trigger workflow CI/CD con primo push
    \item Analisi baseline results e popolamento directory \texttt{analysis/}
    \item Identificazione aree critiche da migliorare
    \item Iterazione con miglioramenti e re-test
\end{enumerate}

\end{lstlisting}

\subsection{Conclusioni Setup}
\label{subsec:setup-conclusions}

Il setup e la configurazione iniziale del progetto sono stati completati con successo. Sono stati configurati:
\begin{itemize}
    \item ✅ Repository GitHub con remote origin e upstream
    \item ✅ File di configurazione per tutti gli strumenti di analisi
    \item ✅ GitHub Actions workflows per CI/CD automatizzata
    \item ✅ Plugin Maven per analisi locale
    \item ✅ Documentazione LaTeX dettagliata per Overleaf
\end{itemize}

Il progetto è ora pronto per:
\begin{enumerate}
    \item Setup account SonarCloud e DockerHub
    \item Prima esecuzione analisi locale per baseline
    \item Trigger workflow CI/CD con push su GitHub
    \item Raccolta e documentazione risultati analisi
    \item Iterazione miglioramenti basati su finding
\end{enumerate}
