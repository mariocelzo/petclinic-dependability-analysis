\section{Introduction}
\label{sec:introduction}

\subsection{Motivation}

Software dependability is a critical aspect of modern software engineering, encompassing 
reliability, availability, safety, security, and maintainability \cite{avizienis2004basic}. 
As software systems become increasingly complex and integral to critical operations, ensuring 
their dependability becomes paramount.

This project aims to conduct a comprehensive dependability analysis of the Spring PetClinic 
application, evaluating various quality dimensions through automated analysis tools and 
systematic testing approaches.

\subsection{Project Goals}

The primary objectives of this analysis are:

\begin{enumerate}
    \item \textbf{Evaluate Code Quality}: Assess the codebase using static analysis tools 
    to identify bugs, vulnerabilities, and code smells.
    
    \item \textbf{Measure Test Coverage}: Determine the extent to which the existing test 
    suite exercises the application code using coverage analysis.
    
    \item \textbf{Assess Test Quality}: Evaluate the effectiveness of the test suite through 
    mutation testing to identify weak spots.
    
    \item \textbf{Analyze Performance}: Establish performance baselines and identify bottlenecks 
    using microbenchmarking techniques.
    
    \item \textbf{Identify Security Vulnerabilities}: Scan for known vulnerabilities in 
    dependencies and security issues in code.
    
    \item \textbf{Improve Test Coverage}: Generate additional tests automatically to fill 
    coverage gaps.
    
    \item \textbf{Document Findings}: Provide comprehensive documentation of analysis results 
    and improvement recommendations.
\end{enumerate}

\subsection{Application Under Analysis}

\textbf{Spring PetClinic} is a sample Spring Boot application designed to demonstrate the 
capabilities of the Spring Framework. It implements a simple veterinary clinic management 
system with the following features:

\begin{itemize}
    \item Owner and pet management
    \item Veterinarian information
    \item Visit scheduling
    \item Pet type categorization
\end{itemize}

The application serves as an excellent subject for dependability analysis due to:

\begin{itemize}
    \item \textbf{Realistic complexity}: Contains typical web application patterns
    \item \textbf{Active maintenance}: Regularly updated by the Spring team
    \item \textbf{Comprehensive features}: Includes database interaction, validation, and MVC architecture
    \item \textbf{Existing test suite}: 44 tests providing baseline for comparison
\end{itemize}

\subsection{Technology Stack}

The application utilizes the following technologies (as analyzed in this project):

\begin{itemize}
    \item \textbf{Framework}: Spring Boot 4.0.0-SNAPSHOT
    \item \textbf{Language}: Java 21 (Eclipse Temurin)
    \item \textbf{Database}: H2 (in-memory for development/testing)
    \item \textbf{Build Tool}: Maven 3.9+ (via Maven Wrapper)
    \item \textbf{Testing}: JUnit 5, Mockito, Spring Boot Test
    \item \textbf{Web}: Spring MVC, Thymeleaf
    \item \textbf{CI/CD}: GitHub Actions
    \item \textbf{Containerization}: Docker (multi-stage build)
    \item \textbf{Code Analysis}: SonarCloud
\end{itemize}

\subsection{Report Structure}

This report is organized as follows:

\begin{description}
    \item[Chapter 2] provides background on dependability concepts and analysis tools
    \item[Chapter 3] describes the methodology employed in this analysis
    \item[Chapter 4] presents detailed analysis results for each evaluation criterion
    \item[Chapter 5] discusses the findings and their implications
    \item[Chapter 6] documents improvements implemented and their impact
    \item[Chapter 7] concludes with lessons learned and future work
\end{description}
