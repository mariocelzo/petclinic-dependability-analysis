\section{Background and Related Work}
\label{sec:background}

\subsection{Software Dependability}

Software dependability is defined by Avizienis et al. \cite{avizienis2004basic} as 
"the ability to deliver service that can justifiably be trusted." It encompasses 
several key attributes:

\begin{description}
    \item[Availability] Readiness for correct service
    \item[Reliability] Continuity of correct service
    \item[Safety] Absence of catastrophic consequences on the user(s) and the environment
    \item[Integrity] Absence of improper system alterations
    \item[Maintainability] Ability to undergo modifications and repairs
\end{description}

\subsection{Analysis Tools Overview}

\subsubsection{Code Coverage Analysis - JaCoCo}

JaCoCo (Java Code Coverage) is a free code coverage library for Java \cite{jacoco}. 
It provides several coverage metrics:

\begin{itemize}
    \item \textbf{Line Coverage}: Percentage of executable lines executed
    \item \textbf{Branch Coverage}: Percentage of branches (if/else) executed
    \item \textbf{Method Coverage}: Percentage of methods invoked
    \item \textbf{Complexity Coverage}: Coverage weighted by cyclomatic complexity
\end{itemize}

\subsubsection{Mutation Testing - PITest}

PITest performs mutation testing by introducing small changes (mutations) to the code 
and verifying if tests detect them \cite{pitest}. A high mutation score indicates 
effective tests.

Mutation operators include:
\begin{itemize}
    \item Conditionals boundary mutator (e.g., $<$ to $\leq$)
    \item Negate conditionals mutator (e.g., $==$ to $\neq$)
    \item Math mutator (e.g., $+$ to $-$)
    \item Return values mutator
\end{itemize}

\subsubsection{Performance Benchmarking - JMH}

Java Microbenchmark Harness (JMH) is a framework for writing, running, and analyzing 
micro-benchmarks \cite{jmh}. It handles:
\begin{itemize}
    \item JVM warmup
    \item JIT compilation effects
    \item Dead code elimination
    \item Garbage collection interference
\end{itemize}

\subsubsection{Automated Test Generation - EvoSuite}

EvoSuite automatically generates JUnit tests using search-based techniques \cite{fraser2011evosuite}. 
It employs genetic algorithms to evolve test suites that maximize coverage.

\subsubsection{Security Analysis - OWASP Tools}

\begin{description}
    \item[Dependency-Check] Identifies known vulnerabilities (CVEs) in project dependencies
    \item[FindSecBugs] SpotBugs plugin for security-specific bug patterns
\end{description}

\subsection{Quality Metrics}

\subsubsection{Code Coverage Thresholds}

Industry standards suggest:
\begin{itemize}
    \item Minimum acceptable: 60-70\%
    \item Good coverage: 80-85\%
    \item Excellent coverage: 90\%+
\end{itemize}

However, Marick \cite{marick1999coverage} notes that high coverage doesn't guarantee 
quality—it's necessary but not sufficient.

\subsubsection{Mutation Score}

A mutation score above 75\% is generally considered good, indicating that the test 
suite successfully detects most injected faults \cite{jia2011analysis}.

\subsection{Related Work}

Previous studies on Spring PetClinic analysis:
\begin{itemize}
    \item [Reference relevant academic papers]
    \item [Reference similar analysis projects]
\end{itemize}

\subsection{OWASP Top 10}

The OWASP Top 10 \cite{owasp2021} represents the most critical security risks to 
web applications. Our security analysis focuses on identifying these risks in the 
PetClinic application.
